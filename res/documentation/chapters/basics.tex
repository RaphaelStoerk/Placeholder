%% introduction.tex
%%

\section{Grundlagen}
\label{ch:basics}

\subsection{Vorgaben}
\label{ch:presets}
Folgende Vorgaben wurden zu Beginn des Semesters an das Tutorium angebracht:
\begin{description}
	\item[Motivation] Das Projekt wird nur verfolgt, wenn die Teilnehmer des Tutoriums dies auch möchten. Hierfür wurde eine entsprechende Umfrage erstellt bei der heraus kam, dass die Mehrzahl der Teilnehmer dieses Projekt angehen möchten.
	\item[Intention] Das Projekt soll gemeinsam in der Zeit des Tutoriums, soweit es die pflichtigen Aufgaben zulassen, entwickelt werden. Die Intention hinter diesem Projekt ist die zusätzliche Bereitstellung von theoretischem und praktischem Wissen im Bereich der Grundlagen der Informatik und Programmierung.
	\item[Themenfreiheit] Die Teilnehmer haben demokratische Stimmgewalt über die Themen und Funktionen des Projektes. Der Tutor behält sich jedoch ein allgemeines Veto-Recht vor, welches in Stichsituationen zum Einsatz kommen kann.
	\item[Erreichbarkeit] Alle im Tutorium entwickelten Ressourcen werden den Teilnehmern online jederzeit zugänglich gemacht. Damit soll erreicht werden, dass die Teilnehmer jedwedem Thematischen Vorwissen die Möglichkeit haben alle erarbeiteten Projektteile zu verstehen und zu diskutieren. Zudem dienen diese Ressourcen als Hilfe für weitere Aufgaben und als zusätzliche Vorbereitung auf die Klausur. Hierbei ist zu beachten, dass keine Musterlösungen oder Plagiate den Weg in dieses Projekt finden dürfen.
\end{description}

\subsection{Projekt-Parameter}
\label{ch:params}
In diesem Abschnitt wird im Verlauf der Projektentwicklung aufgeführt, welche Parameter durch die Teilnehmer erdacht und gewählt wurden. Der bisherige Stand findet sich in folgender Auflistung:
