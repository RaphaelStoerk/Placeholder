%% introduction.tex
%%
\section{Einführung}
\label{ch:introduction}
In Verbindung mit den aktuellen Tutorien in Einführung in die Informatik an der Universität Ulm möchten die Teilnehmer des Tutoriums von Raphael Störk, Mittwoch 16-18, neben den Pflichtübungen einen zusätzlichen Arbeitsaufwand leisten um eine breitere Grundlage und ein größeres Basiswissen für ihr weiteres Studium zu schaffen. Dafür wurde von dem Tutor ein gemeinsames Projekt vorgeschlagen, welches in den Tutorien als Zusatzübung erarbeitet werden soll. Ziel dieses Projektes ist nicht in erster Linie die Produktion einer vollständig entwickelten und getesteten Software sondern die Erlernung Grundlegender Programmierkenntnisse anhand praktischer Beispiele, welche im Verlauf des Projekts erarbeitet werden.

Der grundlegende Gedanke hinter diesem Projekt ist die Anwendung der im Tutorium kennen gelernten und vorgestellten Themen auf ein großes, kontinuierlich erweitertes Projekt um das Verständnis der Studenten in der Hinsicht zu fordern, dass klar wird wofür diese Themen im Bereich der Informatik und Programmierung wichtig sind und verwendet werden.

Klar muss auch sein, dass dieses Projekt eine rein freiwillige Übung für die Studenten ist. Die Teilnahme und Einbringung an dieser Übung wird nicht in die Bepunktung der Übungsvorleistung mit eingerechnet und ist auch niemals als eine Voraussetzung für das Bestehen einer Universitären Leistung gedacht. Auch ist dem Tutor keine weitere Tutoriumsgruppe bekannt, die solch ein Projekt durchzuführen gedenkt. Daher sollte eindeutig gesagt werden, dass dieses Projekt als reine Zusatzübung entwickelt wird. Entsprechend sollte die Richtigkeit und Vollständigkeit mit Bedacht genossen werden.


