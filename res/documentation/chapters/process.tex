%% introduction.tex
%%
\section{Entwicklung}
\label{ch:timeline}
Dieses Kapitel enthält eine kurze Übersicht über die in den einzelnen Übungswochenen erarbeiteten Themen und den damit verbundenen Projektteilen. 
\subsection{Woche 1 - Grundlagen}
\label{ch:weekone}
\line
\begin{center}
	\textbf{KEYWORDS}
	
	 Algorithmen, Zahlenbasen, Horner-Schema, Java Mini-Workflow
\end{center}
\line

In Woche 1 (hier auch Woche 0 miteinbezogen, diese hat kein eigenes Kapitel verdient) wurde als Grundlage für die weiteren Wochen gezeigt, wie eine Kommandozeile zu verwenden ist und wie man mit dieser ein einfaches Java-Programm kompiliert (\textit{übersetzt}) und ausführt. Wichtig dabei sind die Befehle \textbf{javac} zum compilen und \textbf{java} zum Ausführen. Ein Programm in einer Datei namens \textit{Hello.java} würde man also mit folgenden Befehlen ausführen:
\begin{verbatim}
.../> javac Hello.java

.../> java Hello
Hello World!
\end{verbatim}
Außerdem wurde in dieser Woche über Zahlensysteme gesprochen. Es ist bei der Programmierung oft hilfreich und manchmal notwendig das Konzept binärer und anderer Zahlen zu kennen. Eine Zahl im allgemeinen ist eine Aneinanderreihung von Ziffern, die Stelle der Ziffer in dieser Reihung bestimmt deren Gewichtigkeit. 

So ist im Dezimalsystem die Zahl $123_{10}$ zu verstehen als $1 * 10^2 + 2 * 10^1 + 3 * 10^3$. Entsprechend wäre die Zahl $1010_{2}$ im Binärsystem zu verstehen als $1 * 2^3 + 0 * 2^2 + 1 * 2^1 + 0*2^0 = 8 + 2 = 10_{10}$. Die Umrechnung zwischen verschiedenen Systemen kann mittels des \textbf{Horner-Schemas} erreicht werden (vgl. Vorlesung).

Vor allem aber wurde in dieser Woche bereits die Entwicklung von Algorithmen besprochen. Dies geschah über die Schritte \textbf{Problemspezifikation, Problemabstraktion, Algorithmenentwurf, Verifikation und Aufwandsanalyse}. Mit diesen Schritten wurde für das Projekt versuchsweise ein Algorithmus erarbeitet, der einer Punkte-Berechnung am Ende eines Spiels darstellen könnte.
Es entstand folgender Pseudocode:
\pagebreak
\begin{verbatim}
Algorithm "Check Won":

input s, p, c;
c = c + round_down(p / 10);
while c > 1:
  if c % 2 == 0
    c = c / 2;
    s = s + 1;
  else
    c = 0
  end if
end loop

if s > 10
  print 'spieler hat gewonnen'
else
  print 'spieler hat verloren'
end if
\end{verbatim}
Es wurde bestimmt, dass dieser Algorithmus einen Aufwand der Form $log(n)$ aufweist.

Zuletzt wurde auch eine \textbf{Java-Klasse} als Start für unsere Anwendung aufgebaut. Diese enthielt nur eine simple main-Methode und eine Begrüßung des Nutzers:

\begin{verbatim}
"Application.java":

import java.util.Scanner;

public class Application {

  public static void main(String[] args) {
 
    // create a new Scanner that lets us read in input from the user
    Scanner scan = new Scanner(System.in);
    System.out.print("Welcome to <placeholder>, what is your name?");
    String name = scan.nextLine();
    System.out.print("Hello " + name + "!");
    scan.close();
  }
}
\end{verbatim}
\subsection{Woche 2 - }
\label{ch:weektwo}
\line
\begin{center}
	\textbf{KEYWORDS}
	
	Pseudocode, Datentypen, Boolsche Ausdrücke, Menüführung
\end{center}
\line
In der zweiten Übungswoche wurde das Konzept der Algorithmenentwicklung mit Pseudocode wiederholt. Zudem wurden Grundlegende Datentypen in Java kennen gelernt. Die primitiven Datentypen sind dabei: \textbf{byte, short, int, long, char, float, double und boolean}. Als nicht-primitiver Datentyp wurden \textbf{Strings} angesprochen. Es wurde erklärt wie diese Datentypen den zugehörigen Speicher belegen und in welcher Situation welcher Datentyp sinnvoll wäre.

Auch wurde bereits das Konzept Logischer Ausdrücke und die Verwendung von Wahrheitswerten besprochen. Dazu wurde die Funktion von Negation \textbf{!}, logischem und \textbf{\&\&}, logischem oder \textbf{||} und dem Vergleich \textbf{==} erklärt und angewandt.

Mit diesen nun zugänglichen Grundlagen wurde für das Projekt dann gemeinsam ein simples Anwendungsmenü aufgebaut. Es wurde ermittelt, dass ein Nutzer über die Eingabe verschiedener Zahlen entsprechende Punkte des Menüs erreichen kann. Über sinnvolle Abfragen im Programm soll die Eingabe des Nutzers mit den vorgegebenen Punkten abgeglichen werden. Dabei enstand folgendes Menü:

\begin{verbatim}
"Application.java" - Main Method

[...]
  public static void main(String[] args)
    // create a new Scanner that lets us read in input from the user
    Scanner scan = new Scanner(System.in);
    System.out.print("Welcome to <placeholder>, what is your name?");
    String name = scan.nextLine();
    System.out.print("Hello " + name + "!");
    
    // Let user choose an option
    System.out.print("What do you want to do?");
    System.out.print("\t(1) Play Game");
    System.out.print("\t(2) Options");
    System.out.print("\t(3) Loading");
    System.out.print("\n::MENU::> ");
    int opt = scan.nextInt();
    
    // The menu options
    System.out.println("You have entered option: " + opt);
    
    if(opt == 1) {
      System.out.print("This is game");
    } else if(opt == 2) {
      System.out.print("This is options");
    } else if(opt == 3) {
      System.out.print("This is loading");
    } else {
      System.out.print("This is not a valid input");
    }
    scan.close();
  }
[...]

\end{verbatim}
\subsection{Woche 3 - }
\label{ch:weekthree}
\subsection{Woche 4 - }
\label{ch:weekfour}