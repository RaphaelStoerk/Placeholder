\documentclass[12pt,a4paper,titlepage,twoside]{article}
\usepackage[utf8]{inputenc}
\usepackage[T1]{fontenc}
\usepackage[ngerman]{babel}
\usepackage{amsmath}
\usepackage{amsfonts}
\usepackage{amssymb}
\usepackage{graphicx}
\pagestyle{headings}
\def\blankpage{%
	\clearpage%
	\thispagestyle{empty}%
	\addtocounter{page}{-1}%
	\null%
	\clearpage}
\def\line{
	\noindent\makebox[\linewidth]{\rule{\linewidth}{0.4pt}}
}
\begin{document}
	\sffamily
	\begin{titlepage}
		\centering
		\includegraphics[width=0.2\textwidth]{res/unilogo_bild}\par\vspace{1cm}
		{\scshape\LARGE Universität Ulm \par}
		\vspace{1cm}
		{\scshape\Large Freiwilliges EidI Projekt \par}
		\vspace{1.5cm}
		{\huge\bfseries Placeholder\par}
		\vspace{2cm}
		{\Large\itshape Tut Mi 16 2202\par}
		\vfill
		supervised by\par
		Raphael \textsc{Störk}
		
		\vfill
		
		% Bottom of the page
		{\large \today\par}
	\end{titlepage}
\blankpage

	\begin{abstract}
		In diesem Dokument sind alle wichtigen Prozessschritte, Entscheidungen und Ergebnisse dokumentiert, welche im Verlauf des Projektes 'Placeholder' entstanden sind. Das Projekt selber ist eine eigenständige, frewillige Zusatzleistung der Tutoriumsteilnehmer, welches als Unterstützung der Studenten beim Erlernen wichtiger Grundlagen helfen soll. Das Projekt dient insbesondere NICHT als Pflichtteil der Übung sondern ist ein individuelles Zusatzangebot des Tutors.
		
		[Ergebnisse]
	\end{abstract}
\blankpage
\tableofcontents
\blankpage

%% introduction.tex
%%
\section{Einführung}
\label{ch:introduction}
In Verbindung mit den aktuellen Tutorien in Einführung in die Informatik an der Universität Ulm möchten die Teilnehmer des Tutoriums von Raphael Störk, Mittwoch 16-18, neben den Pflichtübungen einen zusätzlichen Arbeitsaufwand leisten um eine breitere Grundlage und ein größeres Basiswissen für ihr weiteres Studium zu schaffen. Dafür wurde von dem Tutor ein gemeinsames Projekt vorgeschlagen, welches in den Tutorien als Zusatzübung erarbeitet werden soll. Ziel dieses Projektes ist nicht in erster Linie die Produktion einer vollständig entwickelten und getesteten Software sondern die Erlernung Grundlegender Programmierkenntnisse anhand praktischer Beispiele, welche im Verlauf des Projekts erarbeitet werden.

Der grundlegende Gedanke hinter diesem Projekt ist die Anwendung der im Tutorium kennen gelernten und vorgestellten Themen auf ein großes, kontinuierlich erweitertes Projekt um das Verständnis der Studenten in der Hinsicht zu fordern, dass klar wird wofür diese Themen im Bereich der Informatik und Programmierung wichtig sind und verwendet werden.

Klar muss auch sein, dass dieses Projekt eine rein freiwillige Übung für die Studenten ist. Die Teilnahme und Einbringung an dieser Übung wird nicht in die Bepunktung der Übungsvorleistung mit eingerechnet und ist auch niemals als eine Voraussetzung für das Bestehen einer Universitären Leistung gedacht. Auch ist dem Tutor keine weitere Tutoriumsgruppe bekannt, die solch ein Projekt durchzuführen gedenkt. Daher sollte eindeutig gesagt werden, dass dieses Projekt als reine Zusatzübung entwickelt wird. Entsprechend sollte die Richtigkeit und Vollständigkeit mit Bedacht genossen werden.



\blankpage
%% introduction.tex
%%

\section{Grundlagen}
\label{ch:basics}

\subsection{Vorgaben}
\label{ch:presets}
Folgende Vorgaben wurden zu Beginn des Semesters an das Tutorium angebracht:
\begin{description}
	\item[Motivation] Das Projekt wird nur verfolgt, wenn die Teilnehmer des Tutoriums dies auch möchten. Hierfür wurde eine entsprechende Umfrage erstellt bei der heraus kam, dass die Mehrzahl der Teilnehmer dieses Projekt angehen möchten.
	\item[Intention] Das Projekt soll gemeinsam in der Zeit des Tutoriums, soweit es die pflichtigen Aufgaben zulassen, entwickelt werden. Die Intention hinter diesem Projekt ist die zusätzliche Bereitstellung von theoretischem und praktischem Wissen im Bereich der Grundlagen der Informatik und Programmierung.
	\item[Themenfreiheit] Die Teilnehmer haben demokratische Stimmgewalt über die Themen und Funktionen des Projektes. Der Tutor behält sich jedoch ein allgemeines Veto-Recht vor, welches in Stichsituationen zum Einsatz kommen kann.
	\item[Erreichbarkeit] Alle im Tutorium entwickelten Ressourcen werden den Teilnehmern online jederzeit zugänglich gemacht. Damit soll erreicht werden, dass die Teilnehmer jedwedem Thematischen Vorwissen die Möglichkeit haben alle erarbeiteten Projektteile zu verstehen und zu diskutieren. Zudem dienen diese Ressourcen als Hilfe für weitere Aufgaben und als zusätzliche Vorbereitung auf die Klausur. Hierbei ist zu beachten, dass keine Musterlösungen oder Plagiate den Weg in dieses Projekt finden dürfen.
\end{description}

\pagebreak

\subsection{Projekt-Parameter}
\label{ch:params}
In diesem Abschnitt wird im Verlauf der Projektentwicklung aufgeführt, welche Parameter durch die Teilnehmer erdacht und gewählt wurden. Der bisherige Stand findet sich in folgender Auflistung:

\subsubsection{Technische Vorgaben}
\label{ch:technical}
\begin{description}
	\item[Sprache] min. Java 8 / Entwickelt mit Java 13
	\item[IDE] Zu Beginn keine, eventuell wird im Verlauf des Semesters auf Eclipse/IntelliJ umgestiegen
	\item[Versionierung] Git, mit Hilfe von GitHub und Gitkraken, für Teilnehmer nur optional
\end{description}
\subsubsection{Genre}
\label{ch:genre}
Noch offen
\subsubsection{Konzept}
\label{ch:concept}
Noch offen
\subsubsection{Story}
\label{ch:story}
Folgende Storyelemente wurden bisher erarbeitet:
\begin{description}
	\item[Zeit] Industrialisierung, 18./19. Jahrhundert
	\item[Ort] London, England
	\item[Lebensraum] Industriegebiet/Armenviertel
	\item[Keywords] Rebellengruppe, Kommunismus, Steampunk, Kriegstreibende Regierung, Drogenkriminalität
	\item[Main Character] Jugendlich/Teenager, Arbeiterfamilie, Indisch/Irische Herkunft, Teil der Rebellengruppe
\end{description}
\blankpage
%% introduction.tex
%%

\section{Anforderungen}
\label{ch:Features}
\subsection{Funktionale Anforderungen}
\label{ch:functional}
\subsection{Qualitäts Anforderungen}
\label{ch:quality}
\blankpage
%% introduction.tex
%%
\section{Entwicklung}
\label{ch:timeline}
\subsection{Woche 1 - Grundlagen}
\label{ch:weekone}
\subsection{Woche 2 - }
\label{ch:weektwo}
\subsection{Woche 3 - }
\label{ch:weekthree}
\subsection{Woche 4 - }
\label{ch:weekfour}
\blankpage
%% introduction.tex
%%
\section{Implementierung}
\label{ch:implementation}
\subsection{Framework}
\label{ch:framework}
\subsection{Architektur}
\label{ch:architecture}
\subsection{Vorstellung der Software}
\label{ch:presentation}
\blankpage
%% introduction.tex
%%
\section{Einordnung}
\label{ch:comparison}
\subsection{Verwandte Arbeiten}
\label{ch:related}
\subsection{Evaluation}
\label{ch:evaluation}
\subsection{Ausblick}
\label{ch:future}


\end{document}